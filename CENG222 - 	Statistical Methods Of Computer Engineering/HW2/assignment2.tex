\documentclass[11pt]{article}
\usepackage[utf8]{inputenc}
\usepackage{float}
\usepackage{amsmath}


\usepackage[hmargin=3cm,vmargin=6.0cm]{geometry}
%\topmargin=0cm
\topmargin=-2cm
\addtolength{\textheight}{6.5cm}
\addtolength{\textwidth}{2.0cm}
%\setlength{\leftmargin}{-5cm}
\setlength{\oddsidemargin}{0.0cm}
\setlength{\evensidemargin}{0.0cm}

\newcommand{\HRule}{\rule{\linewidth}{1mm}}

%misc libraries goes here
\usepackage{tikz}
\usetikzlibrary{automata,positioning}

\begin{document}
\noindent
\HRule
\begin{center}
\Large 
\textbf{CENG 222}  \\
\normalsize 
Assignment 2 \\
Deadline: May 13, 23:59 \\
\end{center}
\begin{flushleft}
\normalsize 
	Full Name:Alper KOCAMAN	\\
	Id Number:2169589
\end{flushleft}
\HRule

% Write your answers below the section tags
\section*{Answer 9.16}
\subsection*{a}

Sample proportion $p\widehat{}$ = $\frac{number\ of\ sampled\ items}{n}$.\\\\
Also,\\$E(p\widehat{})= p$  and Var$(p\widehat{})=\frac{p\times (1-p)}{n}$.\\\\
The paramater of interest $\theta=p_1 - p_2$ and it is estimated by $\theta\widehat{}=p_1\widehat{} - p_2\widehat{}$,this will be center.\\\\
The standard error is estimated by $s(\theta\widehat{}\ )=\sqrt{\frac{p_1\widehat{}\times (1-p_1\widehat{})}{n_1} + \frac{p_2\widehat{}\times (1-p_2\widehat{})}{n_2}}$,this will be margin.\\\\
Thus.confidence interval formula for difference of proportions is:\\

$\theta\widehat{}\mp z_{0.02/2}\times s(\theta\widehat{}\ )=(p_1\widehat{} - p_2\widehat{}) \mp z_{0.02/2} \times \sqrt{\frac{p_1\widehat{}\times (1-p_1\widehat{})}{n_1} + \frac{p_2\widehat{}\times (1-p_2\widehat{})}{n_2}}$.\\\\
In this problem, $n_1 = 250$ and $n_2 = 300$.\\So,$p_1\widehat{}=10/250 =0.04$ and $p_2\widehat{}=18/300=0.06$.\\\\
Center = $0.04-0.06=-0.02$.\\
$z_{0.02/2} = 2.326$\\
Margin = $\sqrt{\frac{0.04\times 0.96}{250} + \frac{0.06\times 0.94}{300}}$.\\\\
Thus,the answer is $0.02 \mp 0.043$ = $(-0.063,0.023)$.	 


\subsection*{b}

Null Hypothesis - $H_0$: There is no significant difference between two lots' quality.\\\\
Alternative Hypothesis - $H_A$: There is a significant difference between two lots' quality.\\\\
Step 1 : Test Statistic.\\

For these Bernoulli data, the variance depends on the unknown parameters $p_1$ and 
$p_2$  which are estimated by the sample proportions $p_1\widehat{}$ and $p_2\widehat{}$.\\\\
$\frac{p_1\widehat{} - p_2\widehat{}}{\sqrt{\frac{p_1\widehat{}\times (1-p_1\widehat{})}{n_1} + \frac{p_2\widehat{}\times (1-p_2\widehat{})}{n_2}}}$ = $\frac{0.04 - 0.06}{\sqrt{\frac{0.04\times 0.96}{250} + \frac{0.06\times 0.94}{300}}}$ = $-1.06$.\\\\
From part a) $z_{0.01} = 2.326$, since\ $|-1.06| < |2.326|$,we accept hypothesis Null ( $H_0$ ).\\
There is no significant difference between two lots' quality.\\

\section*{Answer 10.2}

There are 64 values in question and the mean value of values given in the question is $\bar{X} = 5$.\\\\
Cumulative distribution function is $1-e^{-\lambda x}$  where \\ $\lambda = 1/\bar{X}=1/5=0.2$\\\\So,the equation becomes $F(x)$ = $1-e^{-0.2 x}$.\\\\
Since there are 64 values , these data should be groupped.The bins should contain at least 5 element and number of groups should be between 5-8(preferably).\\
A suitable groupping can be $B_0=[0,2]$ minutes , $B_1=[2,4]$ minutes , $B_2=[4,6]$ minutes, $B_3=[6,8]$ minutes ,$B_4=[8,\infty]$ minutes.There are 5 groups.\\\\
By the aid of above function  $F(x)$ = $1-e^{-0.2 x}$, the values of that partitons calculated as:\\

$x=0$ $\rightarrow $  $1-e^{-0.2 x}$ = 0\\

$x=2$ $\rightarrow $ $1-e^{-0.2 x}$ = 0.329\\

$x=4$ $\rightarrow $ $1-e^{-0.2 x}$ = 0.55\\

$x=6$ $\rightarrow $ $1-e^{-0.2 x}$ =0.70\\

$x=8$ $\rightarrow $ $1-e^{-0.2 x}$ =0.80\\

Then expexted number of elements in bins corresponding to their values :\\

$B_0=[0,2] = 64 \times 0,329$ = $21.05$\\

$B_1=[2,4] = 64 \times (0.55-0,329)$ = $14.14$\\

$B_2=[4,6] = 64 \times (0.7-055)$ = $9.6$\\

$B_3=[6,8] = 64 \times(0.8-0.7)$ = $6.4$\\

$B_4=[8,\infty] = 64 \times (1-0.8)$ = $12.8$\\\\
For each of bin , using formula $\frac{{(obs -exp)}^2}{exp}$:\\

$B_0$ $\rightarrow $ = 3.08\\

$B_1$ $\rightarrow $ = 0.24\\

$B_2$ $\rightarrow $ = 3.04\\

$B_3$ $\rightarrow $ =0.06\\

$B_4$ $\rightarrow $ =0.003\\\\

Then,using formula $\chi^2 = \Sigma \frac{{(obs -exp)}^2}{exp}$:\\

$3.08 + 0.24 + 3.04 + 0.06 + 0.003=6.423$.\\

Degrees of Freedom is 4  and the corresponding value of that is 9.49 with a significance level of \%5(from table of coursebook page 420). Found value is less than 9.49 so there is no evidence against an exponential distribution. 



\section*{Answer 10.3}
\subsection*{a}

There are lots of data(100) and this data should be groupped.There can be 10 bins for these data.\\Mean of these data is $\bar{X}= -0.058$.\\ 

$B_0=[-\infty,-2]$ observed data:4\\

$B_1=[-2,-1.5]$ observed data:4\\

$B_2=[-1.5,-1.0]$ observed data:15\\

$B_3=[-1.0,-0.5]$ observed data:9\\

$B_4=[-0.5,0]$ observed data:22\\

$B_5=[0,0.5]$ observed data:15\\

$B_6=[0.5,1]$ observed data:12\\

$B_7=[1,1.5]$ observed data:11\\

$B_8=[1.5,2]$ observed data:7\\

$B_9=[2,\infty]$ observed data:1\\

From the Standard normal distribution table from book(page 417) expected values calculated as:

$B_0=[-\infty,-2]$ expexted data: $0.0228\times100$ = 2.28\\

$B_1=[-2,-1.5]$ expexted data: $(0.0668 -0.0228) \times100$ = 4.4\\

$B_2=[-1.5,-1.0]$ expexted data: $(0.1587-0.0668)\times100$ = 9.19\\

$B_3=[-1.0,-0.5]$ expexted data: $(0.3085-0.1587)\times100$ = 14.98\\

$B_4=[-0.5,0]$ expexted data: $(0.5-0.3085)\times100$ = 19.15\\

$B_5=[0,0.5]$ expexted data: $(0.6915-0.5)\times100$ = 19.15\\

$B_6=[0.5,1]$ expexted data: $(0.8413-0.6915)\times100$ = 14.98\\

$B_7=[1,1.5]$ expexted data: $(0.9332-0.8413)\times100$ = 9.19\\

$B_8=[1.5,2]$ expexted data: $(0.9772-0.9332)\times100$ = 4.4\\

$B_9=[2,\infty]$ expexted data: $(1-0.9772)\times100$ = 2.28\\\\
For each of bin , using formula $\frac{{(obs -exp)}^2}{exp}$:\\

$B_0=[-\infty,-2]$ value = 1.30\\

$B_1=[-2,-1.5]$ value = 0.04\\

$B_2=[-1.5,-1.0]$ value = 3.67\\

$B_3=[-1.0,-0.5]$ value = 2.39\\

$B_4=[-0.5,0]$ value = 0.42\\

$B_5=[0,0.5]$ value = 0.90\\

$B_6=[0.5,1]$ value = 0.60\\

$B_7=[1,1.5]$ value = 0.36\\

$B_8=[1.5,2]$ value = 1.53\\

$B_9=[2,\infty]$ value = 0.71\\\\

Then,using formula $\chi^2 = \Sigma \frac{{(obs -exp)}^2}{exp}$ = 11.92 .\\\\
Degrees of Freedom is 9  and the corresponding value of that is 16.9 with a significance level of \%5(from table of coursebook page 420). Found value is less than 16.9 so there is no evidence against Standard Normal distribution.


\subsection*{b}

Pdf of Uniform distribution is $\frac{1}{b-a}$.(a= -3 , b= 3)\\

$B_0=[-\infty,-2]$ observed data:4\\

$B_1=[-2,-1.5]$ observed data:4\\

$B_2=[-1.5,-1.0]$ observed data:15\\

$B_3=[-1.0,-0.5]$ observed data:9\\

$B_4=[-0.5,0]$ observed data:22\\

$B_5=[0,0.5]$ observed data:15\\

$B_6=[0.5,1]$ observed data:12\\

$B_7=[1,1.5]$ observed data:11\\

$B_8=[1.5,2]$ observed data:7\\

$B_9=[2,\infty]$ observed data:1\\\\
Expected values calculated as:\\

$B_0=[-\infty,-2]$ expexted data = 16.67\\

$B_1=[-2,-1.5]$ expexted data = 8.33\\

$B_2=[-1.5,-1.0]$ expexted data = 8.33\\

$B_3=[-1.0,-0.5]$ expexted data = 8.33\\

$B_4=[-0.5,0]$ expexted data = 8.33\\

$B_5=[0,0.5]$ expexted data = 8.33\\

$B_6=[0.5,1]$ expexted data = 8.33\\

$B_7=[1,1.5]$ expexted data = 8.33\\

$B_8=[1.5,2]$ expexted data = 8.33\\\\

$B_9=[2,\infty]$ expexted data = 16.67\\\\
For each of bin , using formula $\frac{{(obs -exp)}^2}{exp}$:\\

$B_0=[-\infty,-2]$ value = 9.63\\

$B_1=[-2,-1.5]$ value = 2.25\\

$B_2=[-1.5,-1.0]$ value = 5.33\\

$B_3=[-1.0,-0.5]$ value = 0.05\\

$B_4=[-0.5,0]$ value = 22.41\\

$B_5=[0,0.5]$ value = 5.33\\

$B_6=[0.5,1]$ value = 1.61\\

$B_7=[1,1.5]$ value = 0.85\\

$B_8=[1.5,2]$ value = 0.21\\

$B_9=[2,\infty]$ value = 14.73\\\\

Then,using formula $\chi^2 = \Sigma \frac{{(obs -exp)}^2}{exp}$ = 62.42 .\\\\
Degrees of Freedom is 9  and the corresponding value of that is 16.9 with a significance level of \%5(from table of coursebook page 420). Found value is far more than 16.9 so there is a strong evidence against Uniform distribution.

\subsection*{c}

Since data sample is large ($n \geq 30 $) ,central limit theorem can be applied.So , although this problem is a counterexample , a large data sample can follow both Uniform and Standard distribution at the same time theoretically.\\


\end{document}
