\documentclass[12pt]{article}
\usepackage[utf8]{inputenc}
\usepackage{float}
\usepackage{amsmath}


\usepackage[hmargin=3cm,vmargin=6.0cm]{geometry}
%\topmargin=0cm
\topmargin=-2cm
\addtolength{\textheight}{6.5cm}
\addtolength{\textwidth}{2.0cm}
%\setlength{\leftmargin}{-5cm}
\setlength{\oddsidemargin}{0.0cm}
\setlength{\evensidemargin}{0.0cm}

\newcommand{\HRule}{\rule{\linewidth}{1mm}}

%misc libraries goes here
\usepackage{tikz}
\usetikzlibrary{automata,positioning}

\begin{document}

\noindent
\HRule \\[3mm]
\begin{flushright}

                                         \LARGE \textbf{CENG 222}  \\[4mm]
                                         \Large Statistical Methods for Computer Engineering \\[4mm]
                                        \normalsize      Spring '2017-2018 \\
                                           \Large   Take Home Exam 1 \\
                    \normalsize Deadline: May 25, 23:59 \\
                    \normalsize Submission: via COW
\end{flushright}
\HRule

\section*{Student Information }
%Write your full name and id number between the colon and newline
%Put one empty space character after colon and before newline
Full Name : Alper KOCAMAN \\
Id Number :  2169589\\

% Write your answers below the section tags
\section*{Answer 1}

By using monte carlo size formula;\\\\
$N\geq0.25\times\frac{(z_{\alpha/2})}{\varepsilon}$.\\
By putting values,\\\\
$N\geq0.25\times\frac{1.96}{0.005}$ = $38416$ trials are needed to obtain results that have such precisions.\\\\
After than,it is needed to generate a poisson random variable which has $\lambda=20$,since an expectation of 1 hour is 4 and we are expexted to generate a random number for caught minions in 5 hours.\\\\
After than,an "a" , "b" and "c" values are found in order to apply rejection method.\\\\
After these values are controlled,corresponding "weight" and "speed" values are found randomly for each minion.Then it is investigated that whether it's weight is more than 2 times of its speed.If it is,then a counter is incremented.After inside for loop terminates,if number of minions that have this property is more than 6,we increment another counter to accept this minion set.\\\\
With tha aid of outer for loop,we can do this job for 38,416 times,which gives us a "correct" answer within given precision. 


\section*{Answer 2}

The only thing differs here is not finding sets that have more than 6 minions,we are asked to find the total weight.\\
So , for every minion,it's weight is summed up with total weight of minions.\\
Then,by dividing this total number with 38,416,correct result is obtained.\\

\section*{Answer 3}
The "a" and "b" value is found randomly with matlab functions exprnd() and normrnd().\\\\
The estimation of this value can be found randomly in matlab with "mean" function.I do this job 1000 times to converge the correct result.\\




\end{document}
